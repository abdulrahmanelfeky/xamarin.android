\chapter{مقدمة}

\section {ماهو \textLR{Xamarin} }
هو اداة يمكن من خلالها تطوير تطبيقات على منصات متعددة \textLR{Cross-Platform} بإستخدام لغة \textLR{C\#} وايضا يمكن تطوير تطبيقات اصلية \textLR{Native Apps} 
\section{نبذة عن تاريخ \textLR{Xamarin}}
استوحي اسم \textLR{Xamarin} من قرد \textLR{Tamarin} بإستبدال حرف \textLR{T} بحرف  \textLR{X} لانهم استخدموا في بداية التطوير \textLR{Theme Ximian}
في مايو 1102 قام فريق من شركة \textLR{Mono} بالإعلان عن شركة جديدة تهدف الى تطوير تطبيقات للهواتف وقامت الشركة في عام 2102 بإصدار \textLR{Xamarin.Mac} 
وفي فبراير 3102 تم اصدار  \textLR{Xamarin Studio}
وقامت شركة مايكروسوفت بشراء الشركة بسعر بتروح بين 004 الى 005 مليون دولار 
وسوف تقوم شركة مايكروسوفت بجعل \textLR{Xamarin SDK} مفتوح المصدر

\section{ماذا يقصد بالمنصات المتعددة \textLR{(Cross-Platform)}}
تقنية تسمح بتطوير التطبيق مرة واحدة والنشر على المنصات مختلفة وهي \textLR{Android , iOS , Windows}
\section{ماهي \textLR{(Native Apps)}}
ببساطة هي أن التطبيق الذي ستبنيه باستخدام منصة \textLR{Xamarin} لاستهداف منصة ما سيتم إنتاجه كما لو أنه طور باستخدام أدوات المطورين المخصصة للتطوير لتلك المنصة، إذاً لا فرق في الأداء أو واجهات المستخدم أو الوصول الغير مقيد لوظائف وواجهات النظام البرمجية بين تطبيق تم تطويره لنظام \textLR{Android} باستخدام أداوت المطورين الخاصة بنظام أندرويد ولغة جافا.

